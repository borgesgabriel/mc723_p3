\documentclass[11pt, a4paper]{article}

\usepackage[hmargin=2.5cm,vmargin=2.5cm]{geometry}
\usepackage{amsmath}
\usepackage{listings}
\usepackage[brazil]{babel}
\usepackage[utf8x]{inputenc}
\usepackage{url}
\usepackage{array}
\usepackage{float}
\usepackage{tocloft}
\usepackage{indentfirst}
\usepackage{graphicx,color}
\usepackage{caption}
\usepackage{subcaption}
\usepackage{subfig}
\usepackage{pdfpages}
\usepackage{algorithm}
\usepackage[noend]{algpseudocode}
\usepackage{color}
\usepackage{enumerate}
\usepackage{multirow}
\usepackage{xcolor,colortbl}
\usepackage{longtable}
\usepackage[justification=centering]{caption}
\usepackage[cm]{fullpage}
\usepackage{wrapfig}
\usepackage{amssymb}
\usepackage{titling}
\usepackage{lscape}
\renewcommand{\cftsecleader}{\cftdotfill{\cftdotsep}}

\definecolor{mygray}{rgb}{0.7,0.7,0.7}
\definecolor{mygreen}{rgb}{0,0.6,0}

\lstdefinestyle{mystyle}{
    % choose the language of the code
    language=C,
    % where to put the line-numbers
    numbers=left,
    % the step between two line-number
    stepnumber=1,
    % how far the line-numbers are from the code
    numbersep=5pt,
    % choose the background color.
    %backgroundcolor=\color{mygray},
    % show spaces adding particular underscores
    showspaces=false,
    % underline spaces within strings
    showstringspaces=false,
    % show tabs within strings adding particular underscores
    showtabs=false,
    % sets default tabsize to x spaces
    tabsize=2,
    % sets the caption-position to bottom
    captionpos=b,
    % sets automatic line breaking
    breaklines=true,
    % sets if automatic breaks should only happen at whitespace
    breakatwhitespace=true,
    % show the filename of files included with \lstinputlisting;
    %title=\lstname,
    %commentstyle=\color{blue},
    %stringstyle=\color{magenta},
    %keywordstyle=\color{brown},
    %identifierstyle=\color{blue},
    %numberstyle=\color{black},
    frame=tb,
    aboveskip=3mm,
    belowskip=3mm,
    columns=flexible,
    basicstyle={\small\ttfamily},
    numberstyle=\tiny\color{black},
    keywordstyle=\color{blue},
    commentstyle=\color{mygray},
    stringstyle=\color{red},
    escapeinside={<@}{@>}
}
\lstset{style=mystyle}

\begin{document}

\begin{titlepage}

\title{\vspace*{4cm}\textbf{Projeto $3^{+}$ \\ Processador Multicore}\vspace{1cm}}
\author{Gabriel Borges \\ RA: 116909 \\ Grupo 5}
\date{30 de junho de 2015}
\clearpage\maketitle

\tableofcontents
\thispagestyle{empty}

\vspace*{\fill}

\begin{center}
Professor: Rodolfo Jardim de Azevedo \\
MC723 - Laboratório de Projeto de Sistemas Computacionais \\
Turmas A e B
\end{center}

\end{titlepage}

\newpage

\section{Introdução}

O objetivo deste projeto é desenvolver e estender o funcionamento de um processador de múltiplos núcleos. Para isso, se utilizou o simulador ArchC (uma linguagem de descrição de arquiteturas de processadores baseada na linguagem SystemC) para simular a execução de um programa relevante na arquitetura de processador MIPS, com o intuito de observar o funcionamento da arquitetura \textit{multicore}, bem como de constatar os ganhos atingidos com o módulo de \textit{harware} criado para acelerar o desempenho do sistema. Finalmente, foi também necessário implementar em \textit{hardware} recursos que permitissem a execução concorrente de trechos críticos de código.

\section{Programa de Testes}

O programa de testes utilizado tinha o objetivo de calcular, com a melhor precisão possível, uma aproximação para o número $\pi$. Para tanto, nos valemos das seguintes relações: \\
\begin{equation}
\pi = 4 * atan(1)
\end{equation}
\\
\begin{equation}
atan(x) = \int_{0}^{x} \frac{1}{1+x^2} dx \rightarrow \pi = 4 * \int_{0}^{1} \frac{1}{1+x^2} dx  
\end{equation}

O método usa somas de \textit{Riemann} com intervalos definidos pelo usuário para fazer o cálculo da aproximação para a integral. Em sua versão \textit{multithread}, o programa subdivide essas tarefas entre os processos, somando os subcálculos para chegar no valor final ao término da execução.

O código foi adaptado para certas especificidades do projeto de \url{http://cs.calvin.edu/curriculum/cs/374/homework/threads/01/pi.c}.

\section{Decisões de projeto}

Para corretamente projetar e analisar a arquitetura \textit{multicore} do processador, consideramos os seguintes fatores:

\subsection{Controlador de processadores}

Como já evidenciado no roteiro inicial deste projeto e empregado no sistema apresentado, é comum que, num regime \textit{multicore} de processadores, assim que houver \textit{boot} no sistema ou \textit{hard reset}, os núcleos sejam inicialmente inicializados e colocados num estado à espera do sistema operacional. O \textit{bootstrap processor} (processador a carregar o sistema operacional em primeira instância) então começa a executar instruções, que eventualmente farão uso dos demais processadores. Quando isso ocorre, este ou estes sai(em) do modo de espera e é(são) colocado(s) à executar instruções definidas.

Neste projeto, a plataforma de intercâmbio entre os núcleos \textit{per se} e o sistema operacional é um controlador de processadores, representado por um periférico chamado \textit{cores\_controller}. A comunicação entre o periférico e o sistema é feita através de três interfaces:

\begin{itemize}
\item A função \textbf{\texttt{number\_of\_cores}} retorna a quantidade de núcleos do processador disponíveis ao sistema operacional;
\item A função \textbf{\texttt{is\_core\_on}} recebe um índice de núcleo do processador (esses índices são naturais de $0$ a \texttt{number\_of\_cores()}$ - 1$), e retorna se o núcleo explicitado está ou não ativo (ou ``ligado'');
\item A função \textbf{\texttt{set\_core}} recebe um \textit{status} (\textit{on/off}) e um índice de núcleo do processador, e associa ao processador correspondente àquele índice o \textit{status} recebido. Essa função é utilizada para ``ligar'' e ``desligar'' núcleos quando se fizer necessário.
\end{itemize}

Em termos de implementação na arquitetura \textit{ArchC}, o controlador foi feito de maneira análoga à memória fornecida no repositório original do projeto\footnote{disponível em \url{/home/staff/rodolfo/mc723/base_platform.git}.}. Desta sorte, seu funcionamento pode ser resumidamente explicado da seguinte forma:
\begin{enumerate}
\item Requisições tratadas pelo \textit{bus} através de seu método \texttt{transport} que outrora eram inevitavelmente direcionadas ao \texttt{transport} equivalente na memória agora podem ser levadas ao do referido controlador, de acordo com a posição do endereço acessado;
\item Na lógica do controlador, essas requisições podem ser tanto de leitura (caso das funções \textbf{\texttt{number\_of\_cores}} e \textbf{\texttt{is\_core\_on}}) como de gravação (caso da função \textbf{\texttt{set\_core}}). Outros mecanismos de controle são empregados para determinar qual a função desejada, bem como seus argumentos;
\item A requisição é atendida (especificamente, nos módulos disponíveis em \texttt{cores\_controller.cpp}) e os efeitos colaterais, realizados (caso de ser necessário que núcleos sejam ligados/desligados);
\item Para o sistema operacional, a maneira de acessar tais funcionalidades é de ler de (caso de requisições de leitura) ou gravar em (caso de requisições de gravação) endereços específicos em memória. Essas solicitações são então levadas até o controlador de processadores, no processo descrito acima.
\end{enumerate}

\subsection{Controle de concorrência}

O controle de concorrência foi implementado em \textit{hardware}. Com o intuito de manter a coerência com operações da arquitetura \textit{MIPS}, as instruções implementadas foram as seguintes:
\begin{itemize}
\item \textbf{Load Linked} (\texttt{ll}): Instrução do tipo I, de \textit{opcode} $30_{hex}$. \\ Faz a operação \texttt{R[rt] = M[R[rs]+SignExtImm]} (idêntica à instrução \texttt{lw}), e que seta um campo especial com o valor \texttt{R[rt]} (doravante denominado \texttt{history});
\item \textbf{Store Contiditional} (\texttt{sc}): Instrução do tipo I, de \textit{opcode} $38_{hex}$. Faz as seguintes operações: \\ \textit{a)} \texttt{M[R[rs]+SignExtImm] = R[rt]}; \textit{b)} \texttt{R[rt] = R[Rt] == history ? 1 : 0}. \\ A operação \textit{``a)''} é idêntica à \texttt{sw}, e o item \textit{``b)''} é o que determina se a operação é atômica, registrando essa condição em \texttt{R[rt]}.
\end{itemize}

Com as instruções \texttt{ll} e \texttt{sc} disponíveis em \textit{hardware}, podemos implementar as operações da biblioteca \texttt{pthread} utilizadas no \textit{software} de teste. As funções implementadas são as seguintes:
\begin{itemize}
\item \texttt{pthread\_create}: cria uma nova \textit{thread}, se possível, e destina um processador a trabalhar nela. Também recebe o ponteiro para uma função -- a execução do processador designado deve iniciar nessa função. Sua implementação verifica se existe processador ocioso, e em caso afirmativo, associa a ele a \textit{thread} requisitada;
\item \texttt{pthread\_join}: aguarda até que a \textit{thread} correspondente tenha terminado sua execução (se isto ainda não tiver ocorrido), interrompendo a execução da \textit{thread} atual até que a condição se concretize. Sua implementação interrompe a execução até que o processador correspondente esteja ocioso;
\item \texttt{pthread\_mutex\_init}: inicializa o \textit{mutex} correspondente, permitindo que ele seja bloqueado ou desbloqueado;
\item \texttt{pthread\_mutex\_lock}: operação de bloqueio do \textit{mutex} (aguarda até que o \textit{mutex} esteja liberado, quando o bloqueia e faz uso do recurso crítico). Implementado com o seguinte loop: \\ \texttt{while (load\_linked(mutex) || !store\_conditional(mutex, 1));}
\item \texttt{pthread\_mutex\_unlock}: operação de desbloqueio do \textit{mutex}. Meramente altera o valor da chave para zero.
\end{itemize}

\subsection{\textit{Hardware Offloading}}

Ao analisar a estrutura do código principal de testes, \texttt{pi.c}, é evidente perceber que o gargalo na execução está no loop principal da função \texttt{computePI()}. Portanto, essa foi a seção em que se fez o \textit{hardware offloading}.

Para tanto, criamos mais um componente do dispositivo, de maneira similar à criação do controlador de processadores, descrito acima. A função desse dispositivo seria de efetuar o cálculo da aproximação para $\pi$, para uma \textit{thread} específica, de acordo com o número de \textit{threads} total e o número de intervalos de cálculo. 

Foi criada a função \texttt{offload\_arctan()} com esse objetivo. Se o \textit{offloading} não está ativo, a computação segue em software conforme esperado. Em caso contrário, cada \textit{thread} escreve em um arquivo os parâmetros necessários para o cálculo da função desejada e chama uma operação de leitura, que sinaliza ao sistema no \textit{hardware} que o arquivo gravado está disponível para leitura, e que a \textit{thread} está aguardando a resposta. O componente em \textit{hardware}, por sua vez, lê os argumentos do arquivo, efetua o cálculo desejado e grava o resultado noutro arquivo, que é, por fim, lido novamente pelo programa principal.

\section {Resultados}

É uma propriedade da soma de \textit{Riemann} que quanto maior o número de intervalos utilizados para aproximar uma integral, maior será a precisão da aproximação. Também é esperado que quanto maior o número de processadores utilizados para o desempenho da tarefa, mais rápida deve ser sua execução, descontado o \textit{overhead} de criação e sincronização dos processos. Temos a seguinte tabela com os resultados experimentados:

\begin{table}[h]
\centering
\caption{Performance do programa de testes}
\label{my-label}
\begin{tabular}{|lllll|}
\hline
Offload & T & I      & E    & Tempo gasto (s) \\ \hline
Não     & 1 & 100    & 5.6  & 0.54            \\
Sim     & 1 & 100    & 5.6  & 1.67            \\
Não     & 1 & 1000   & 7.6  & 4.93            \\
Sim     & 1 & 1000   & 7.6  & 1.64            \\
Não     & 2 & 1000   & 7.6  & 3.76            \\
Não     & 7 & 1000   & 7.6  & 2.87            \\
Não     & 7 & 5000   & 9.0  & 14.09           \\
Sim     & 1 & 5000   & 9.0  & 1.65            \\
Sim     & 1 & $10^5$ & 11.6 & 1.67            \\
Sim     & 1 & $10^6$ & 13.6 & 1.67            \\
Sim     & 1 & $10^7$ & 15.6 & 1.83            \\
Sim     & 1 & $10^8$ & 15.7 & 3.40            \\
Sim     & 2 & $10^8$ & 15.7 & 5.78            \\
Sim     & 1 & $10^9$ & 15.3 & 19.03           \\ \hline
\end{tabular}
\end{table}

Legenda:
\begin{itemize}
\item \textbf{Offload}: Determina se o \textit{hardware offload} foi utilizado; 
\item \textbf{T}: Número de \textit{threads} utilizadas no cálculo. Esse número deve variar entre $1$ e $7$, pois uma thread é utilizada para a sincronização dos dados;
\item \textbf{I}: Número de intervalos;
\item \textbf{E}: $-log_{10}(\pi_{medido}/\pi)$, aproximado para uma casa decimal. Quanto maior este valor, mais precisa é a aproximação;
\item \textbf{Tempo gasto (s)}: Tempo ``real'' medido em segundos, aproximado para duas casas decimais.
\end{itemize}

\clearpage

Os resultados nos permitem tomar as seguintes conclusões:

\begin{itemize}
\item Há um certo significativo \textit{overhead} para todos os modos. Na versão sem \textit{offload}, ele começa a perder importância a partir de $I \approx 1000$. Já na com offload, ele parece deixar de se manifestar para $I \approx 10^8$;
\item A influência da paralelização no modo sem \textit{offload} não é diretamente proporcional à redução do tempo gasto: de fato, os programas são executados em menos tempo, mas o fator de redução é bem menor que o de aumento no número de núcleos;
\item A introdução do \textit{offload} permite reduções extremamente significativas nos tempos de execução do programa;
\item Como o hardware de \textit{offload} não é paralelizado, não é vantajoso paralelizar a execução do programa de testes em modo paralelo (já que todo o seu \textit{hot spot} foi transportado para o hardware).
\end{itemize}

\section{Conclusão}

Os objetivos do projeto foram plenamente alcançados: o controlador de processadores está completamente funcional, o controle de concorrência foi implementado segundo as especificações e a implementação de código em \textit{hardware} foi realizada. As metas atingidas nos permitiram fazer as seguintes observações:
\begin{itemize}
\item Ficou bastante clara uma simples e eficiente maneira de inicializar processadores \textit{multicore}, recorrendo à criação de um periférico para o gerenciamento das chamadas de suporte aos núcleos, e encarregando o sistema operacional de definir os momentos de inicialização dos núcleos além do principal (\textit{bootstrap processor});
\item Uma etapa anterior deste projeto envolveu uma implementação das rotinas de \texttt{lock} e \texttt{unlock} em \textit{software}, sem recorrer a instruções atômicas. Embora funcional, a técnica era bastante lenta e com alto \textit{overhead}. Ficava evidente a necessidade da implementação desse recurso em \textit{hardware}. Uma vez tendo se alcançado esse objetivo, pudemos perceber a mudança em performance das rotinas de teste. Finalmente, uma vez que fizemos \textit{wrappers} para funções da biblioteca \texttt{pthread}, transportar outras rotinas de teste que também utilizem a biblioteca para o sistema \textit{MIPS} desenvolvido deve ser pouco trabalhoso, se requerir algum trabalho;
\item A implementação dos ``gargalos'' da rotina em \textit{hardware} nos permitiu observar ganhos consideráveis no desempenho da aplicação -- devido aos longos tempos de espera na configuração \textit{multithread} sem \textit{offload}, a comparação entre os modos não é necessariamente estatisticamente relevante. Se extrapolando os resultados observados, no entanto, notamos ganhos entre $8x$ e $148,000x$. Uma vez não tendo o processador sido concebido para realizar as operações características do ``gargalo'', a concepção de um componente em \textit{hardware} com essa função efetivamente elimina o \textit{overhead} experimentado e torna a computação extremamente rápida. Exemplos de dispositivos reais que fazem uso dessa estratégia são \textit{GPUs}: essas resolvem uma área de problemas que \textit{CPUs} são muito menos eficientes em resolver. Isso se deve à diferença arquitetural dessas unidades de hardware.

\end{itemize}

\textbf{Link para o repositório com o código fonte do projeto: \url{https://github.com/borgesgabriel/mc723_p3}.}

% \bibliography{main}
% \bibliographystyle{plain}


\end{document}
